\documentclass{article}
\usepackage{units}
\usepackage{booktabs}
\usepackage{amsmath, amsthm, amssymb, bm}
\usepackage{tikz, pgfplots}
\usetikzlibrary{shapes, arrows, positioning, fit, calc}
\newtheorem{theorem}{Theorem}
\newtheorem{Property}{Property}
\theoremstyle{remark}
\newtheorem*{defn}{Definition}
\renewcommand{\vec}[1]{\underline{#1}}
\usepackage{units}
\usepackage{fancyvrb}
\fvset{fontsize=\normalsize}

\title{AI1103 Assignement 1}
\author{Hritik Sarkar}

\begin{document}

\maketitle

Question 52. Consider the function $ f(x) $ defined as $ f(x) = c e^{-x^{4}}, x\in \mathbb{R}$. For what value of c is $f$ a probability density function?
\\
\\ a) $\frac{2}{\Gamma (\frac{1}{4})}$
\\ b) $\frac{4}{\Gamma (\frac{1}{4})}$
\\ c) $\frac{3}{\Gamma (\frac{1}{3})}$
\\ d) $\frac{1}{4\Gamma(4) }$
\\

Answer: Gamma function $(\Gamma)$ is defined as 
\[
\Gamma (n) = \int_{0}^{\infty} e^{-x} x^{n-1} dx
\]
For a function $f(x)$ to be a probability density function the below condition has to be satisfied
\[
\int_{-\infty}^{\infty} f(x) dx = 1
\]

We can check the condition for the $f(x)$ given.

\begin{align*}
    I &= \int_{-\infty}^{\infty} c e^{-x^{4}} dx \\
    &= 2c\int_{0}^{\infty}e^{-x^{4}} dx &&\textit{($-x^4$ is an even function)} \\
    &= \frac{2c}{4}\int_{0}^{\infty} e^{-y} y^{-\frac{3}{4}} dy &&\textit{(change of variables)}
\end{align*}
consider
\begin{align*}
  y&=x^4 \\
  \llap{$\Rightarrow$\hspace{50pt}} dy&=4x^3dx\\
  \llap{$\Rightarrow$\hspace{50pt}} dx &= \frac{1}{4} y^{-\frac{3}{4}} dy
\end{align*}
Continuing from above
\begin{align*}
    I &=\frac{c}{2}\int_{0}^{\infty} e^{-y} y^{-\frac{3}{4}} dy \\
    &= \frac{c}{2}\int_{0}^{\infty} e^{-y} y^{(\frac{1}{4}-1)} dy \\
    &= \frac{c}{2} \Gamma (\frac{1}{4}) &&\textit{(comparing with the definition of gamma function)}
\end{align*}

\begin{align*}
    \frac{c}{2}\Gamma (\frac{1}{4})&=1 \\
    \llap{$\Rightarrow$\hspace{50pt}}c&=\frac{2}{\Gamma (\frac{1}{4})} &&\textit{Ans.}
\end{align*}

\end{document}